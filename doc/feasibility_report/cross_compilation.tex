% 这部分是放在交叉编译的开头的,引出下面对scons、cargo和meson的介绍

\subsection{交叉编译的实现}
\indent 交叉编译\cite{Cross_compiler}是指在一个架构的机器上(称为构建机器,\textit{build machine},例如 x86_64)编译出可供另一个不同架构的机器(称为目标机器,\textit{host/target machine},例如 ARM)运行的可执行代码。其本质是使用专门的工具链(编译器、链接器等)生成与目标架构兼容的二进制文件,而不需要在目标架构的硬件上直接编译。交叉编译器对于从一个开发主机编译多个平台的代码非常有用,尤其在目标平台计算资源有限(如嵌入式系统)或开发环境复杂时,直接在目标平台编译可能不可行。

\indent 在我们小组的项目中,我们希望实现一个C语言和Rust语言混合编程的系统,并最终编译出适用于ARM架构的可执行文件,最终在STM32芯片上进行测试。然而,为了实现这一目标,我们首先需要一个高效的构建系统,以便将原本用 C 语言编写的代码与我们用Rust语言改写的部分进行编译和链接,并最终生成可用于ARM平台的可执行文件。

\indent 在选择合适的构建工具时,我们调研了几种流行的构建系统\cite{comparison},包括SCons、Cargo和Meson。它们各自具备不同的特性和适用场景。下面我们将对这三种工具进行比较,并分别测试其在 C 语言与 Rust 语言混合编程场景下的可行性。

\indent 在接下来的部分,我们将详细介绍如何使用这三种工具实现C和Rust代码的交叉编译。