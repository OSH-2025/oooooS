\documentclass{ctexart}
\usepackage{graphicx}
\usepackage{hyperref}
\usepackage{amsmath}
\usepackage{listings, listings-rust}
\usepackage[dvipsnames]{xcolor}

\begin{document}
\subsection{RT-Thread 内存管理}
RT-Thread 提供了多种内存管理方式,以适应不同应用需求:
1. 动态内存堆管理:通过堆管理器在运行时动态分配和释放内存,支持小内存管理、slab 管理和 memheap 管理算法。
2. 静态内存池管理:预先分配固定大小的内存块,提高分配效率并减少碎片化。

\noindent \textbf{具体管理方式}

\noindent
1. 小内存管理算法:通过维护一个包含数据头的链表来管理内存块。数
据头包含 magic 字段(用于检测非法访问)和 used 字段(标记当前
内存块是否被分配)。\\
2. Slab管理算法:基于对象大小划分多个区(zone),每个 zone 包含多个 slab,每个 slab 由多个固定大小的内存块组成。这种方法减少了内存碎片,提高了分配效率。\\
3. Memheap管理算法:允许多个地址不连续的内存堆联合管理,提高了系统的灵活性。在分配内存时,系统会优先从默认内存堆分配,若不足则从其它堆中分配。\\
4. 静态内存池管理:在初始化时分配一块固定大小的内存,并将其划分为多个相同大小的块。线程在内存不足时可以挂起,等待可用内存块。\\
\subsubsection{使用Rust进行改写分析}
\paragraph{优势}
Rust 语言的内存安全特性可以减少传统 C 语言实现中的内存泄漏和悬空指针问题,同时 Rust 的类型系统和借用检查器可以提供编译时安全保障。

\begin{lstlisting}[
    language=C,
    caption={RT-Thread 传统 C 语言内存管理示例},
    basicstyle=\ttfamily\small,
    keywordstyle=\color{blue},
    commentstyle=\color{green},
    stringstyle=\color{red},
    tabsize=2,
    frame=lines,
    captionpos=b
]
void *rt_malloc(rt_size_t size) {
    struct mem_block *block = find_free_block(size);
    if (block) {
        block->used = 1;
        return (void *)(block + 1);
    }
    return NULL;
}
\end{lstlisting}

\noindent 上述 C 语言实现的 `rt\_malloc` 函数首先调用 `find\_free\_block(size)` 查找适合的空闲内存块。如果找到,则将其 `used` 标记为 1,并返回指向该内存块的指针。如果找不到合适的块,则返回 `NULL`,表示分配失败。然而,这种实现依赖手动管理`used` 状态,容易导致内存泄漏或悬空指针问题。

\begin{lstlisting}[
    language=rust,
    caption={Rust 版本的内存管理示例},
    basicstyle=\ttfamily\small,
    keywordstyle=\color{blue},
    commentstyle=\color{green},
    stringstyle=\color{red},
    tabsize=2,
    frame=lines,
    captionpos=b
]
struct MemBlock {
    size: usize,
    used: bool,
}

fn allocate(size: usize) -> Option<&'static mut MemBlock>
{
    let block = find_free_block(size)?;
    block.used = true;
    Some(block)
}
\end{lstlisting}

Rust 版本的 `allocate` 函数利用 `Option` 处理分配失败的情况,避免 `NULL` 指针。`find\_free\_block(size)?` 使用 `?` 操作符,若找不到合适的内存块,则自动返回 `None`,减少显式错误处理代码。此外,Rust 的所有权系统能确保 `block` 在生命周期内不会被误用,从根本上杜绝 use-after-free 错误。

Rust 的主要优势体现在以下几个方面。首先,内存安全性是其最突出的特点。Rust 的所有权系统能够有效防止 C 语言中常见的野指针和 use-after-free 错误,使得内存管理更加可靠。其次,Rust 在多线程环境下提供了更高的线程安全性。其无锁并发数据结构能够减少线程间的竞争,提高系统稳定性。此外,Rust 具有良好的模块化和可维护性。相比于 C 语言,Rust 更加强调代码的可读性和抽象能力,使得开发和维护更加高效。

\paragraph{挑战}
将RT-Thread内存管理用Rust改写面临以下挑战:实时性要求高,Rust 默认的内存管理方式(如 Rc、Arc)可能引入非确定性延迟。需要兼容 C API,RT-Thread 的大部分 API 是用C语言编写的,Rust 需要与现有 C 代码进行交互。嵌入式资源占用少,但是Rust运行时相比C 可能占用更多的 ROM 和 RAM,可能不适合资源极为有限的设备。

\paragraph{可能的解决方案}
为减少 Rust 的额外开销并兼容 RT-Thread 的实时性需求,可考虑:使用 Rust 语言改写核心数据结构,如内存块管理。采用 \textbf{无锁并发} 方案,替换标准 Mutex 以避免优先级反转。同时可以考虑仅在高风险模块(如动态内存管理)使用 Rust,而保留性能关键部分的 C 代码。

\end{document}
