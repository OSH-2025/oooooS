\section{立项依据}

\subsection{问题定义}

\textbf{核心命题}:以Rust语言重构RT-Thread Nano内核,构建兼具\textbf{安全性、实时性和开发效率}的嵌入式实时操作系统。

\subsection{重构范围}

\textbf{核心重构对象}:

\begin{itemize}
    \item 任务调度器(Scheduler):抢占式多任务调度
    \item 进程间通信(IPC):动态内存分配与碎片优化
    \item 时钟管理(Timer):信号量/消息队列同步机制
    \item 内存管理(Heap Allocator):硬件定时器与软件定时器
\end{itemize}

\subsection{技术可行性}

\subsubsection{核心模块的Rust化改造}
\b{Rust 调用 C 代码}
\begin{itemize}
    \item 使用 \texttt{bindgen} 自动生成 RT-Thread 内核 API 的 Rust 绑定(如 \texttt{rt\_thread\_create → unsafe extern "C"})
    \item 案例参考:Linux 内核模块通过 \texttt{\#[repr(C)]} 实现结构体对齐,已验证跨语言兼容性
\end{itemize}

\b{C 调用 Rust 代码}
\begin{itemize}
    \item 对安全抽象层(如内存分配器)使用 \texttt{\#[no\_mangle]} 暴露接口,通过 \texttt{cbindgen} 生成 C 头文件
    \item 性能优化:在中断处理函数中采用 \texttt{\#[inline(always)]} 避免堆栈切换开销
\end{itemize}

\subsubsection{开发-调试工具链}

\textbf{编译环境}:

基于PlatformIO构建多语言工程。

\href{https://community.platformio.org/t/support-for-different-languages-and-compilers/921}{\texttt{PlatformIO Community}} 提出,PlatformIO可以自定义开发平台,其底层基于\textbf{SCons},而SCons支持集成Rust。

\textbf{仿真环境}:

\textbf{Wokwi}:Wokwi可通过配置\texttt{wokwi.toml}与\texttt{diagram.json}来模拟多种开发板与外设,方便调试与验证。

\textbf{调试与验证}:

PlatformIO可直接管理与真实设备的连接,并方便上板调试运行以测试。
\begin{itemize}
    \item 首阶段部署至 STM32F103C8T6 等多型号开发板(我们团队成员至少有4块不同型号的开发板),验证内核基本功能
    \item 性能对比:与原版 C 内核进行基准测试(CoreMark 得分、任务切换耗时)
\end{itemize}

\subsubsection{原项目支持}

\begin{itemize}
    \item 代码可移植性:RT-Thread Nano 内核代码量约 1.2 万行,模块化设计清晰(如 \texttt{kservice.c} 独立于硬件抽象层)
    \item 社区支持:已有 Rust 嵌入式社区(如 \texttt{embedded-hal})提供 GPIO/UART 驱动参考
\end{itemize}

\subsection{关键点}

\begin{itemize}
    \item Rust+C编译环境的搭建
    \item RT-Thread Nano的内核结构与具体实现
    \item Rust改写
    \item 改写成果的调试验证与优化
\end{itemize}

\subsection{预期目标}

\begin{itemize}
    \item \textbf{内存安全}:通过所有权模型和借用检查,消除CWE-119(缓冲区溢出)、CWE-416(释放后使用)等漏洞。
    \item \textbf{并发安全}:Rust的\texttt{Send}/\texttt{Sync} Trait静态验证数据竞争,结合\texttt{Mutex<RefCell<T>>}智能锁,降低调度器数据竞争发生率。
    \item \textbf{性能优化}:尝试优化该系统的性能,如实时性等。
    \item \textbf{精简代码}:利用Rust的优质特性精简代码。
\end{itemize}