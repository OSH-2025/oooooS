\section{立项依据}

\subsection{问题定义}

\textbf{核心命题}:以Rust语言重构RT-Thread Nano内核,构建兼具\textbf{安全性、实时性和开发效率}的嵌入式实时操作系统。

\subsection{重构范围与边界}

\textbf{核心重构对象}:

\begin{itemize}
    \item 任务调度器(Scheduler)
    \item 进程间通信(IPC)
    \item 时钟管理(Timer)
    \item 内存管理(Heap Allocator)
\end{itemize}

\subsection{技术可行性}

\subsubsection{核心模块的Rust化改造}

\begin{itemize}
    \item \textbf{Rust调用C代码}:\\使用\texttt{bindgen}工具自动生成RT-Thread内核API的Rust绑定(FFI),通过\texttt{extern "C"}声明保留原有C实现的模块,作为过渡阶段的兼容层。
    \item \textbf{C调用Rust代码}:\\对Rust重构的安全抽象层(如内存分配器、智能锁),通过\texttt{cbindgen}生成C头文件,确保原有C代码(如HAL驱动)可无缝调用。
\end{itemize}


\subsubsection{开发-调试工具链}

\textbf{编译环境}:

基于PlatformIO构建多语言工程。

\href{https://community.platformio.org/t/support-for-different-languages-and-compilers/921}{\texttt{PlatformIO Community}} 提出,PlatformIO可以自定义开发平台,其底层基于\textbf{SCons},而SCons支持集成Rust。

\subsubsection{仿真验证}

\textbf{Wokwi}:配置\texttt{.wokwi.toml}模拟STM32硬件外设,实时可视化线程状态切换(如通过GPIO电平模拟调度器行为)。


\subsubsection{上板验证}

PlatformIO可直接管理与真实设备的连接,并方便上板调试运行以测试。

\subsubsection{原项目}

RT-Thread完全开源,且拥有丰富完整的文档以及相关生态。

\subsection{关键点}

\begin{itemize}
    \item Rust+C编译环境的搭建
    \item RT-Thread Nano的内核结构与具体实现
    \item Rust改写
    \item 改写成果的调试验证与优化
\end{itemize}

\subsection{预期目标}

\begin{itemize}
    \item \textbf{内存安全}:通过所有权模型和借用检查,消除CWE-119(缓冲区溢出)、CWE-416(释放后使用)等漏洞。
    \item \textbf{并发安全}:Rust的\texttt{Send}/\texttt{Sync} Trait静态验证数据竞争,结合\texttt{Mutex<RefCell<T>>}智能锁,降低调度器数据竞争发生率。
    \item \textbf{性能优化}:尝试优化该系统的性能,如实时性等。
    \item \textbf{精简代码}:利用Rust的优质特性精简代码。
\end{itemize}